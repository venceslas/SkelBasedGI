% ---------------------------------------------------------------------
% EG author guidelines plus sample file for EG publication using LaTeX2e input
% D.Fellner, v1.16, Jan 21, 2009


\title[Skeleton Global Illumination]%
      {Using Skeleton for Global Illumination Computation}

% for anonymous conference submission please enter your SUBMISSION ID
% instead of the author's name (and leave the affiliation blank) !!
\author[V. Biri \& J. Chaussard]
       {V. Biri$^{1}$
        and J. Chaussard$^{2}$
        \\
% For Computer Graphics Forum: Please use the abbreviation of your first name.
         $^1$UPE MLV, LIGM, 5 bd Descartes, France\\
         $^2$Mines Paris Tech, France
%        $^2$ Another Department to illustrate the use in papers from authors
%             with different affiliations
       }

% ------------------------------------------------------------------------

% if the Editors-in-Chief have given you the data, you may uncomment
% the following five lines and insert it here
%
% \volume{27}   % the volume in which the issue will be published;
% \issue{1}     % the issue number of the publication
% \pStartPage{1}      % set starting page


%-------------------------------------------------------------------------
\begin{document}

\maketitle

\begin{abstract}
Abstract where we resume the work
\begin{classification} % according to http://www.acm.org/class/1998/
\CCScat{Computer Graphics}{I.3.3}{Picture/Image Generation}{Line and curve generation}
\end{classification}

\end{abstract}





%-------------------------------------------------------------------------
\section{Introduction}

Lot of bla bla about : 
power and interest of path tracing, 
noise generated by this kind of method,
not adequat for very dark area,
random generation of ray not pertinent,

Our contribution :
* An importance sampling that, finally, takes into account Li
* Ray generation adequate/pertinent in dark areas
* A method to mix between "standard" importance sampling and illumination importance sampling
* Usable for any ray tracing method using importance sampling

%-------------------------------------------------------------------------
\section{Previous work}

%-------------------------------------------------------------------------
\subsection{On global illumination}

Details of research on global illumination and especially all techniques that use importance sampling as for example Pure Path Tracing, Photon Mapping, BiDirectional Ray Tracing, Metropolis Light Transport.

%-------------------------------------------------------------------------
\subsection{On importance sampling}

Details of research done on importance sampling in ray tracing

\subsection{On skeletonization}

Details of traditionnal method of skeletonization including critical kernels

%-------------------------------------------------------------------------
\section{Method overview}

Whole overview of our method

%-------------------------------------------------------------------------
\section{Voxelization}

What can we use for voxelization. 
NOTE : this section could be included in the previous one since it is not a detailled point of our technique

%-------------------------------------------------------------------------
\section{Skeletonization}

Details of how we adapt (John's thesis) "classical" work on critical kernels in our problem. Characteristic 

%-------------------------------------------------------------------------
\section{Light Importance Sampling}

Details of how we use the previously computed information to perform our lighting based importance sampling

\subsection{The short path computation, and the importance direction}

How we compute the short path and the related importance direction.

\subsection{Probability distribution function used}

The function used with importance direction to perform importance sampling

\subsection{Multiple Importance Sampling}

How we can mix light importance sampling to BRDF sampling

%-------------------------------------------------------------------------
\section{Results and discussion}

Works with any ray tracing engine. Works with multiple light source. Works with any kind of BRDF.

Embree used for comparison purpose

Images between pure path tracing and our technique. Same with photon mapping or Bi Directional Ray Tracing. 

Adding cost in rendering time. Lower cost for the same quality. Image showing the resulting amelioration in noise in darkest area.

Discussion : 
* Effect of each parameters
* Mix between blending color and importance driven color
* 
%\begin{figure}[htb]
%   % an empty figure just consisting of the caption lines
%   \caption{\label{fig:ex1}
%     'Empty' figure only to serve as an example of long caption requiring 
%     more than one line. It is not typed centered but aligned on both sides.}
%\end{figure}

%
%%%%
%%%% Figure 1
%%%%
%\begin{figure}[htb]
%  \centering
%  % the following command controls the width of the embedded PS file
%  % (relative to the width of the current column)
%  \includegraphics[width=.8\linewidth]{sampleFig}
%  % replacing the above command with the one below will explicitly set
%  % the bounding box of the PS figure to the rectangle (xl,yl),(xh,yh).
%  % It will also prevent LaTeX from reading the PS file to determine
%  % the bounding box (i.e., it will speed up the compilation process)
%  % \includegraphics[width=.95\linewidth, bb=39 696 126 756]{sampleFig}
%  %
%  \parbox[t]{.9\columnwidth}{\relax
%           For all figures please keep in mind that you \textbf{must not}
%           use images with transparent background! 
%           }
%  %
%  \caption{\label{fig:firstExample}
%           Here is a sample figure.}
%\end{figure}

%-------------------------------------------------------------------------
\section{Conclusion}

Conclusion and future works to be done

%-------------------------------------------------------------------------

\bibliographystyle{eg-alpha}

\bibliography{egbibsample}

%-------------------------------------------------------------------------
\newpage
%
%
%\begin{figure*}[tcb]
%  \centering
%  \mbox{} \hfill
%  % the following command controls the width of the embedded PS file
%  % (relative to the width of the current column)
%  \includegraphics[width=.3\linewidth]{sampleFig}
%  % replacing the above command with the one below will explicitly set
%  % the bounding box of the PS figure to the rectangle (xl,yl),(xh,yh).
%  % It will also prevent LaTeX from reading the PS file to determine
%  % the bounding box (i.e., it will speed up the compilation process)
%  % \includegraphics[width=.3\linewidth, bb=39 696 126 756]{sampleFig}
%  \hfill
%  \includegraphics[width=.3\linewidth]{sampleFig}
%  \hfill \mbox{}
%  \caption{\label{fig:ex3}%
%           For publications with color tables (i.e., publications not offering
%           color throughout the paper) please \textbf{observe}: 
%           for the printed version -- and ONLY for the printed
%           version -- color figures have to be placed in the last page.
%           \newline
%           For the electronic version, which will be converted to PDF before
%           making it available electronically, the color images should be
%           embedded within the document. Optionally, other multimedia
%           material may be attached to the electronic version. }
%\end{figure*}

\end{document}
